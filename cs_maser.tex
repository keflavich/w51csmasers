\documentclass[twocolumn]{aastex62}
\input{preface}
\newcommand{\nraojansky}{\affiliation{\it{Jansky fellow of the National Radio Astronomy Observatory, Socorro, NM 87801 USA }}}

\begin{document}
\title{CS masers in W51 e2e}
\titlerunning{CS masers}
\authorrunning{Ginsburg}

\author[0000-0001-6431-9633]{Adam Ginsburg}
\nraojansky


\begin{abstract}
We report the discovery of maser emission in the two lowest rotational
transitions of CS toward the high-mass protostar W51 e2e with ALMA and the
JVLA.  The masers from CS J=1-0 and J=2-1 are neither spatially nor spectrally
coincident (they are separated by $\sim150$ AU and $\sim30$ \kms), but both
appear to come from the base of the blueshifted outflow from this source.
\end{abstract}

\section{Introduction}

Masers can be powerful tracers of gas motion on very small scales.
They have been observed to trace disks orbiting high-mass stars and black
holes, in outflow shocks, and in low-velocity shocks in the ISM.

The rotational transitions of CS have been theorized to mase
in some environments \citep{Schoenberg1988a}, but they have
never previously been observed as masers.  \citet{Schoenberg1988a}
used an expanding shell model in which temperature, density, and
velocity are all decreasing with radius corresponding to some
mass loss rate and terminal wind velocity. They predict masing
in the CS 1-0 and 2-1 lines under different conditions, though the
masing is fairly weak (factors of a few).  In these models,
the stellar infrared radiation drives the maser.

We report the first detection of maser emission from CS in both
the J=1-0 and J=2-1 transitions.

\section{Observations}
\label{sec:observations}
We report observations from three different observing programs:
ALMA 2013.1.01596.S \citep{Goddi2018a} observed band 6 (1 mm), 2017.1.00293.S
observed Band 3 (3 mm), and VLA 16B-202 observed VLA Q-band (7 mm).
The ALMA data were take in long-baseline configurations, and the VLA data
were taken in the most extended A array.

For the ALMA data, we use the pipeline-produced calibrated data for the
emission line maps.  The Band 6 (1 mm) continuum data were self-calibrated
as described in \citet{Goddi2018a}.  The VLA continuum data were calibrated
with the EVLA pipeline with RFI flagging and hanning smoothing disabled.
The continuum data were self-calibrated with 9 iterations of phase-only
corrections.  The line cubes were made using the pipeline-calibrated data.

We imaged the CS J=2-1 98 GHz line with robust 0.5 weighting with 3 \kms
channels, resulting in a beam $0.067\arcsec\times0.043\arcsec$.  The CS J=1-0
49 GHz line was included in a continuum band with 6 \kms resolution, and we
imaged it with robust 0 weighting, resulting in a beam
$0.043\arcsec\times0.037\arcsec$.
The SiO J=5-4 and J=2-1 lines and band 6 (1 mm) continuum were imaged with
robust 0.5 weighting.

The VLA and ALMA data used the same phase calibrator, J1922+1530, with coordinates
that differ by 0.003\arcsec mostly in RA.  Both the VLA and ALMA positions are
offset from the SIMBAD position by about 0.0015 \arcsec.  The VLA measurement
set data incorrectly report the stored coordinate for J1922+1530 as being in
the FK5 system, while the ALMA measurement sets correctly report the stored
coordinate as being in ICRS.  The difference between the ICRS and FK5 system is
about 0.03\arcsec at this position, close to the beam size, and therefore is
highly relevant for these data.  We corrected the images for the coordinate system
offset (0.03\arcsec), but did not correct for the $\approx0.003\arcsec$ discrepancy in
calibrator position, so our systematic pointing error is approximately
0.003\arcsec (16 AU).
% VLA:
%   1    NONE J1922+1530          19:22:34.699188 +15.30.10.03213 J2000   1       14747525
% ALMA:
%    2    none J1922+1530          19:22:34.699385 +15.30.10.03177 ICRS    2       16900560

%NOPE! no pointing offset afeter all.
% We measure an offset between the VLA and ALMA images using the
% \texttt{image-registration} toolkit to cross-correlate the images of W51 e2w.
% W51 e2w is a bright, extended, highly structured HII region bright at all
% observed wavelengths and therefore provides a high target to cross-calibrate
% on.  The total measured offset is 0.029\arcsec, which is less than a beam in
% all of the data sets used here but much larger than the expected relative
% positional accuracy within the images.  The pointing uncertainty should
% therefore be regarded as at least 0.029\arcsec.  We have corrected the VLA
% Q-band images to match the ALMA images, assuming the ALMA data have more
% reliable absolute positions.

\section{Results}
We report the detection of two transitions of CS, J=2-1 and J=1-0, with
peak brightness $T_{B,max}\approx7000$ K.
These lines peak at different locations both spatially (Figure
\ref{fig:overlay}) and spectrally (Figure \ref{fig:spectra}).
We report Gaussian fit parameters to the line profiles and to the peak intensity
images in Table \ref{tab:linepars}; the errors reported in the table are fit
errors assuming no correlation between the parameters, and they do not include
the systematic pointing accuracy problem noted in Section \ref{sec:observations}.

The CS J=2-1 line is observed in the blueshifted component of the
previously-detected SiO outflow \citep{Goddi2018a}.
It has both a compact component, which we will refer to as the CS 2-1 maser,
and an extended component that traces the inner envelope of the blueshifted
outflow.  This extended component can be seen more clearly in the 42 \kms
channel of the channel maps (Figure \ref{fig:channelmaps}).  The maser
peak is located at the base of the blueshifted outflow (Figure
\ref{fig:overlay}) at a velocity of $v_{lsr}=34.50\pm0.07$ \kms.

\begin{figure*}[htp]
    \includegraphics[width=\textwidth]{figures/W51e2e_sio_outflow_with_CS_contours.pdf}
    \caption{An image of the SiO 5-4 outflow (red and blue corresponding to
    red and blueshifted emission integrated over 74-118 \kms and -32 to 55
    \kms, respectively) with continuum shown in green.  The contours show SiO
    2-1 in red (0.05, 0.1 K \kms) and blue (0.1, 0.2 K \kms), the CS 2-1
    maser in white (0.1, 0.2, 0.5, 0.8 K \kms integrated over -32 to 55 \kms),
    and the CS 1-0 maser in black (2000, 4000, 8000 K peak intensity over
    the range 55-74 \kms).  }
    \label{fig:overlay}
\end{figure*}

The CS J=1-0 line is only seen as a single spatial component centered
at approximately $v_{lsr}=64 \pm 6$ \kms.  It is centered closer to the
central continuum source, but still slightly toward the blueshifted outflow.
The CS J=1-0 and J=2-1 masers are offset by $0.036\arcsec \pm 0.12$\arcsec
($190\pm60$ AU).

\begin{table*}[htp]
\centering
\caption{Line Fit Parameters}
\begin{tabular}{llllll}
    \label{tab:observations}
Line Name & Amplitude & $v_{LSR}$ & $\sigma_{FWHM}$ & RA (ICRS) & Dec (ICRS) \\
          &         K &      \kms &            \kms & $\deg$    & $\deg$ \\
\hline
CS J=1-0 &      6800                  &$      64 \pm        6$     &$       7.3 \pm        1.2$ &$    290.9331902 \pm       0.0000025$ &$     14.5095795 \pm       0.0000021$ \\
CS J=2-1 &$      6700 \pm        200$ &$     34.50 \pm       0.07$ &$      5.27 \pm       0.17$ &$    290.9331982 \pm       0.0000025$ &$     14.5095852 \pm       0.0000021$ \\

\hline
\end{tabular}
\label{tab:linepars}
\par
Because of the coarse spectral resolution, the amplitude of the CS 1-0 line is
poorly constrained, and we report its fitted width and centroid assuming a
fixed amplitude of 6800 K.
\end{table*}


\begin{figure}[htp]
\includegraphics[width=0.45\textwidth]{figures/CS1-0_maser_JyandK.pdf}
\includegraphics[width=0.45\textwidth]{figures/CS2-1_maser_JyandK.pdf}
\caption{Plots of the (a) CS 1-0 and (b) CS 2-1 spectra toward their
respective peaks.}
\label{fig:spectra}
\end{figure}


\section{Analysis}
The peak brightness of the observed lines is $\sim6800$ K at $\sim0.07\arcsec$
(350 AU at 3 mm) and $\sim0.04\arcsec$ (200 AU at 7 mm) resolution.  It is unlikely that the
molecules are in thermal equilibrium at $T_K \geq 6000$ K, since such high
temperatures would more likely result in dissociation of the molecules
\citep[e.g.,][]{Pattillo2018a}.
%TODO: check that it would actually dissociate

At the observed spectral resolution of 3 \kms and 6 \kms, the CS J=2-1 line is
marginally resolved with FWHM=5.25 \kms and the CS J=1-0 line is unresolved.
Since the inferred line width and the observed spectral resolution are both
much greater than the thermal width for CS at 6000 K, $v_{therm}\approx1~\kms$,
we cannot conclude anything about the intrinsic line width.

While both transitions have a consistent peak brightness of $\sim6800$ K,
which is below that typical of maser transitions in other molecules, these
emission peaks are at different velocities and may be from different spatial
locations (separated
by about 190 AU), and therefore they are not emitted by the same material.
This velocity difference rules out a thermal origin.

If we assume the masers trace orbiting material, their velocity and spatial
separation can be used to infer the central source mass.  At a separation of 30
\kms and 190 AU, assuming the masers come from opposite ends of a disk such
that $v_{orb}=15\pm1$~\kms and $R_{disk}=95\pm30$~AU, the implied contained mass is
$M=24_{-10}^{+12}$~\msun.  While these assumptions are not particularly
justified, the mass is of order that suggested by \citet{Ginsburg2017a} and
\citet{Goddi2018a} based on luminosity estimates.


Our data also include the W51 IRS2/north and W51e8 HMYSOs
\citep{Ginsburg2017a}.  No compact and bright CS emission sources were detected
toward any other HMYSO candidates in the region, with a peak CS 2-1 flux limit
${S_{98 \mathrm{GHz}} \leq 15 \mathrm{mJy~\perbeam}}$, an order of magnitude
fainter than the peaks seen in W51e2e.  However, plentiful extended CS 2-1
emission is seen around each of the HMYSOs.  This difference indicates that
there is something unique about the chemistry, geometry, or excitation in the
blueshifted outflow lobe of W51e2e.

%It is possible that a region within a few AU of the the central MYSO is heated
%to such high temperatures.  To provide a high enough column density of CS
%to exhibit the observed brightness, this region may have developed an
%equilibrium between CS formation and destruction.

\begin{figure*}
    \includegraphics[]{figures/CS_maser_channel_maps.pdf}
    \caption{Channel maps of the CS J=2-1 transition.  Channels are 3 \kms
    wide.  Darker colors indicate higher intensity.  Contours are overlaid at
    [0.02,0.04,0.06,0.08,0.1,0.125] Jy/beam.  Absorption against the continuum
    is evident from 50-70 \kms.  The redshifted outflow is faint but still
    detected at
    $\sim72.5\kms$.
    }
    \label{fig:channelmaps}
\end{figure*}

\section{Conclusions}
We have detected two emission lines from the J=2-1 and J=1-0 transitions of CS
with high brightness temperature indicating that these lines are masers.
This is the first reported detection of maser emission from CS.
While the HMYSO W51e2e exhibits these masers, many neighboring HMYSOs that
are similar in luminosity, core mass, and apparent evolutionary stage
do not.  The presence of two CS masers from different rotational states
at different velocities and locations in the source suggest that there is something
unique about this object's radiation field, geometry, or chemistry that
promotes CS maser formation.

These CS masers join a growing list of rarely-detected maser transitions
that may trace a unique phase in the formation of high-mass protostars.
Like the NH$_3$, H$_2$CO, and SiO masers, there are only 1-10 known sources of
each of these masers.  Other HMYSOs should be searched for masers in these
transitions to determine how common they are.

\textbf{Acknowledgements}
I thank Todd Hunter for providing literature references on CS masers and Lorant
Sjouwerman for discussion about a lack of CS masers in AGB stars.

\appendix
\section{Velocity check}
We carefully checked the velocity measurements in the VLA data, since an offset
of a few channels is comparable to the Earth's motion and we are using a
continuum band to infer velocity information.  The observations we analyze were
take on December 26 and 30, 2016 and January 7, 2017.  On these dates, the
topocentric-to-barycentric velocity correction is -9.0, -7.4, and -4.1 \kms,
respectively.  The barycentric-to-LSR correction is 16.5 \kms.  The observed CS
1-0 peak is in channel 24 (zero-indexed) of spectral window 26, which has
channel 0 at  48957.667, 48957.932, and 48958.478 MHz, respectively, for each
of the three observations.  The channel width is 1 MHz.  The LSR velocity of
this channel is at 64.3 \kms on all three dates.  This value is consistent with
our reported velocity of $v_{LSR}(\mathrm{CS~J=}1-0) = 65.5 \pm 0.7 \kms$.

No such sanity check is needed for the ALMA data, since the channel maps clearly
show the outflow at appropriate velocities with morphology that matches the SiO
outflow seen at both 1 mm and 3mm.

\section{Variability check}
We also checked for variability in the CS 1-0 maser, and found no evidence for variability
across four observing epochs (2016-10-02, 2016-12-26, 2016-12-30, 2017-01-07; the
former was not included in our merged data set because of bad calibration).
The position remained constant and the flux remained consistent to within single-epoch
observing errors.

\input{solobib}
\end{document}
