\documentclass[twocolumn]{aastex62}
\input{preface}

\begin{document}
We report the discovery of maser emission in the two lowest transitions of CS.

\section{Observations}
We report observations from three different observing programs....

We measure an offset between the VLA and ALMA images using
the \texttt{image-registration} toolkit to cross-correlate the images
of e2.  The total measured offset is 0.029\arcsec, which is less than
a beam in all of the data sets used here.  The pointing uncertainty should
therefore be regarded as at least 0.029\arcsec.  We have corrected
the VLA Q-band images to match the ALMA images, assuming the ALMA
data have more reliable absolute positions.

\section{Results}
We report the detection of two transitions of CS, J=2-1 and J=1-0, with
peak brightness $T_{B,max}\approx6000$ K.
These lines peak at different locations both spatially (Figure
\ref{fig:overlay}) and spectrally (Figure \ref{fig:spectra}).

The CS J=2-1 line is observed in the blueshifted component of the outflow.
It has both a compact component, which we will refer to as the CS 2-1 maser,
and an extended component that traces the inner envelope of the blueshifted
outflow.  This extended component can be seen more clearly in the 42 \kms
channel of the channel maps (Figure \ref{fig:channelmaps}).  The maser
peak is located at the base of the blueshifted outflow (Figure
\ref{fig:overlay}) at a velocity of $v_{lsr}=34.50\pm0.07$ \kms.

\begin{figure*}[htp]
    \includegraphics[width=\textwidth]{figures/W51e2e_sio_outflow_with_CS_contours.pdf}
    \caption{An image of the SiO 5-4 outflow (red and blue corresponding to
    red and blueshifted emission integrated over 74-118 \kms and -32 to 55
    \kms, respectively) with continuum shown in green.  The contours show SiO
    2-1 in red (0.05, 0.1 K \kms) and blue (0.1, 0.2 K \kms), the CS 2-1
    maser in white (0.1, 0.2, 0.5, 0.8 K \kms integrated over -32 to 55 \kms),
    and the CS 1-0 maser in black (2000, 4000, 8000 K peak intensity over
    the range 55-74 \kms).  }
    \label{fig:overlay}
\end{figure*}

The CS J=1-0 line is only seen as a single spatial component centered
at approximately $v_{lsr}=64 \pm 6$ \kms.  It is centered closer to the
central continuum source, but still slightly toward the blueshifted outflow.
The CS J=1-0 and J=2-1 masers are offset by 0.03\arcsec (150 AU), which is
marginally significant given the absolute positional offsets.

\begin{table*}[htp]
\centering
\caption{Line Fit Parameters}
\begin{tabular}{llllll}
    \label{tab:observations}
Line Name & Amplitude & $v_{LSR}$ & $\sigma_{FWHM}$ & RA   & Dec \\
          &         K &      \kms &            \kms & \deg & \deg \\
\hline
CS J=1-0 &      6800                  &$      64 \pm        6$     &$       7.3 \pm        1.2$ &$    290.9331902 \pm       0.0000025$ &$     14.5095795 \pm       0.0000021$ \\
CS J=2-1 &$      6700 \pm        200$ &$     34.50 \pm       0.07$ &$      5.27 \pm       0.17$ &$    290.9331982 \pm       0.0000025$ &$     14.5095852 \pm       0.0000021$ \\

\hline
\end{tabular}
\par
Because of the coarse spectral resolution, the amplitude of the CS 1-0 line is
poorly constrained, and we report its fitted width and centroid assuming a
fixed amplitude of 6800 K.
\end{table*}


\begin{figure}[htp]
\includegraphics[width=0.45\textwidth]{figures/CS1-0_maser_JyandK.pdf}
\includegraphics[width=0.45\textwidth]{figures/CS2-1_maser_JyandK.pdf}
\caption{Plots of the (a) CS 1-0 and (b) CS 2-1 spectra toward their
respective peaks at XX and YY.}
\label{fig:spectra}
\end{figure}


\section{Analysis}
The peak brightness of the observed lines is $\sim6800$ K at $\sim0.07\arcsec$
(350 AU at 3 mm) and $\sim0.04\arcsec$ (200 AU at 7 mm) resolution.  It is unlikely that the
molecules are in thermal equilibrium at $T_K \geq 6000$ K, since such high
temperatures would more likely result in dissociation of the molecules
\citep{Black?}.

At the observed spectral resolution of 3 \kms and 6 \kms, the CS J=2-1 line is
marginally resolved with FWHM=5.25 \kms and the CS J=1-0 line is unresolved.
Since the inferred line width and the observed spectral resolution are both
much greater than the thermal width for CS at 6000 K, $v_{therm}\approx1~\kms$,
we cannot conclude anything about the intrinsic line width.

While both transitions have a consistent peak brightness of $\sim6800$ K,
which is not typical of maser transitions, these emission peaks are at
different velocities and may be from different spatial locations (separated
by about 150 AU), and therefore they are not emitted by the same material.
This velocity difference rules out a thermal origin.


%It is possible that a region within a few AU of the the central MYSO is heated
%to such high temperatures.  To provide a high enough column density of CS
%to exhibit the observed brightness, this region may have developed an
%equilibrium between CS formation and destruction.

\begin{figure*}
    \includegraphics[]{figures/CS_maser_channel_maps.pdf}
    \caption{Channel maps of the CS J=2-1 transition.  Channels are 3 \kms
    wide.  Darker colors indicate higher intensity.  Contours are overlaid at
    [0.02,0.04,0.06,0.08,0.1,0.125] Jy/beam.  Absorption is evident from 50-70
    \kms.  The redshifted outflow is faint but still detected at
    $\sim72.5\kms$.
    }
    \label{fig:channelmaps}
\end{figure*}

\section{Conclusions}

\input{solobib}
\end{document}
